\subsection{Discussion}
\label{sec:discussion}

\paragraph{Embodied vs. Abstract Reasoning.}
Our results demonstrate that embodied reasoning requires distinct computational mechanisms from abstract reasoning in current language models. The persistent performance gaps across reasoning-specialized architectures indicate that chain-of-thought approaches cannot bridge the representational divide between symbolic manipulation and physical constraint processing. Current transformer architectures lack the specialized components necessary for grounding abstract representations in continuous physical properties.

\paragraph{Architectural Limitations.}
The constraint selection failures reveal that current attention mechanisms cannot dynamically filter task-relevant physical constraints from environmental noise. Unlike abstract reasoning tasks where all provided information typically bears relevance, embodied scenarios require selective attention over spatially and temporally distributed constraint sets. The discrete scaling transitions at 7B parameters indicate that embodied reasoning demands sufficient working memory capacity to simultaneously track environmental states, capability constraints, and coordination requirements—a computational bottleneck absent in pure language tasks.

\paragraph{Limitations and Future Work.}
Our text-based framework abstracts away continuous control, sensorimotor feedback, and real-time constraints present in physical embodied systems. While this abstraction enables systematic evaluation, it may not capture all aspects of embodied intelligence. The identified architectural requirements require validation in continuous control settings. Future work should investigate how these components integrate with sensorimotor processing and examine whether the observed computational bottlenecks persist in physically grounded systems. Additionally, exploring hybrid symbolic-neural architectures that can explicitly reason about physical laws while maintaining learned flexibility represents a promising direction\citep{rabinowitz2018machine}.