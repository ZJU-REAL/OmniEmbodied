\clearpage
% \appendix
\section*{Appendix}

\subsection{Benchmark Statistics and Coverage}

\label{sec:dataset_statics}

\benchmark encompasses 1,500 scenarios with 64,057 interactive objects, providing comprehensive coverage across diverse domains and task complexities. Tables~\ref{tab:dataset_overview} through \ref{tab:domain_distribution} present detailed statistics demonstrating the scale and diversity of our benchmark.

\begin{table}[htbp]
\centering
\footnotesize
\begin{minipage}[t]{0.47\textwidth}
\centering
\begin{tabular}{lr}
\toprule
\textbf{Metric} & \textbf{Count} \\
\midrule
Total Scenarios & 1,500 \\
Total Task Files & 1,481 \\
Total Task Instances & 16,592 \\
Interactive Objects & 64,057 \\
Spatial Nodes (Rooms) & 6,634 \\
Average Objects per Scene & 42.7 \\
Average Rooms per Scene & 4.4 \\
Collaborative Agent Pairs & 1,481 \\
\bottomrule
\end{tabular}
\caption{Dataset scale and composition.}
\label{tab:dataset_overview}
\end{minipage}
\hfill
\begin{minipage}[t]{0.47\textwidth}
\centering
\begin{tabular}{lrr}
\toprule
\textbf{Task Category} & \textbf{Count} & \textbf{\%} \\
\midrule
\multicolumn{3}{l}{\textit{Single-Agent (65\%)}} \\
Direct Command & 2,684 & 16.2 \\
Attribute Reasoning & 2,669 & 16.1 \\
Tool Use & 2,190 & 13.2 \\
Compound Reasoning & 2,214 & 13.3 \\
\midrule
\multicolumn{3}{l}{\textit{Multi-Agent (35\%)}} \\
Explicit Collaboration & 2,160 & 13.0 \\
Implicit Collaboration & 2,582 & 15.6 \\
Compound Collaboration & 2,093 & 12.6 \\
\midrule
\textbf{Total} & \textbf{16,592} & \textbf{100} \\
\bottomrule
\end{tabular}
\caption{Task category distribution.}
\label{tab:task_distribution}
\end{minipage}
\end{table}

\vspace{0.3cm}

\begin{table}[htbp]
\centering
\footnotesize
\begin{minipage}[t]{0.47\textwidth}
\centering
\begin{tabular}{lrr}
\toprule
\textbf{Category/Material} & \textbf{Count} & \textbf{\%} \\
\midrule
\multicolumn{3}{l}{\textit{Object Categories}} \\
Container & 17,632 & 27.5 \\
Tool & 15,134 & 23.6 \\
Appliance & 8,963 & 14.0 \\
Furniture & 6,234 & 9.7 \\
Consumable & 4,890 & 7.6 \\
Others & 11,204 & 17.6 \\
\midrule
\multicolumn{3}{l}{\textit{Material Types (Top 10 of 1,123)}} \\
Plastic & 13,767 & 21.5 \\
Metal & 11,274 & 17.6 \\
Wood & 8,263 & 12.9 \\
Glass & 6,277 & 9.8 \\
Fabric & 5,060 & 7.9 \\
Ceramic & 3,843 & 6.0 \\
Silicon & 1,794 & 2.8 \\
Aluminum & 1,601 & 2.5 \\
Steel & 1,153 & 1.8 \\
Others & 11,025 & 17.2 \\
\bottomrule
\end{tabular}
\caption{Object categories and material distribution.}
\label{tab:object_materials}
\end{minipage}
\hfill
\begin{minipage}[t]{0.47\textwidth}
\centering
\begin{tabular}{lrr}
\toprule
\textbf{Domain/Room Type} & \textbf{Count} & \textbf{\%} \\
\midrule
\multicolumn{3}{l}{\textit{Application Domains}} \\
Laboratory & 585 & 39.0 \\
Office & 282 & 18.8 \\
Industrial & 173 & 11.5 \\
Medical & 93 & 6.2 \\
Household & 93 & 6.2 \\
Educational & 63 & 4.2 \\
Retail & 48 & 3.2 \\
Service & 30 & 2.0 \\
Entertainment & 27 & 1.8 \\
Transportation & 23 & 1.5 \\
Others & 83 & 5.6 \\
\midrule
\multicolumn{3}{l}{\textit{Room Types (Top 5)}} \\
Laboratory & 1,876 & 28.3 \\
Storage & 1,234 & 18.6 \\
Workspace & 987 & 14.9 \\
Office & 765 & 11.5 \\
Workshop & 543 & 8.2 \\
\bottomrule
\end{tabular}
\caption{Domain and spatial distribution.}
\label{tab:domain_distribution}
\end{minipage}
\end{table}

\paragraph{Physical Property Modeling.}
The benchmark features exceptional attribute diversity with 6,381 distinct property types. Core physical properties are comprehensively modeled: weight (64,047 objects), material composition (35,411 objects), size dimensions (22,820 objects), color (28,034 objects), and dynamic states (17,547 objects). This rich attribute space enables sophisticated reasoning about physical constraints and object affordances.

\paragraph{Action Space and Tool Ecosystem.}
The framework supports 214 distinct action types, partitioned into basic actions (60\%) available to all agents and tool-dependent actions (40\%) requiring specific capabilities. Among the 64,057 objects, 15,134 are classified as tools (23.6\%), with 13,482 objects possessing the \texttt{provides\_abilities} attribute that enables dynamic capability extension. This design enables realistic modeling of how agents acquire new abilities through tool use.

\paragraph{Cross-Domain Coverage.}
The benchmark spans diverse application domains, with laboratory environments comprising 39.0\% of scenarios, followed by office (18.8\%), industrial (11.5\%), and medical (6.2\%) settings. This distribution reflects our emphasis on professional environments where embodied reasoning is particularly critical. Each domain presents unique challenges: laboratory settings require precise tool usage and material handling, office environments emphasize multi-agent coordination, and industrial scenarios demand reasoning about heavy equipment and safety constraints.

\paragraph{Quality Assurance and Expert Trajectories.}
All 16,592 task instances include expert demonstration trajectories averaging 8.7 steps, providing optimal solutions for comparison and learning. Each trajectory undergoes validation to ensure physical feasibility and task completion. The evaluation framework supports multi-level verification including spatial relationships (1,300 location checks), state transitions (open/closed, on/off states), and compound conditions for complex task assessment. This comprehensive validation ensures that all tasks are both challenging and solvable, maintaining benchmark integrity while achieving unprecedented scale.


\subsection{Analysis}

\paragraph{Failure Mode Analysis.}
Systematic failure analysis reveals task-specific performance bottlenecks that vary distinctly across model scales. Tool Use failures are dominated by exploration deficits (31.2\%), where models fail to locate required tools while maintaining spatial representations. Models below 7B parameters exhibit 2.7-fold higher failure rates (84.2\% vs. 31.2\%), confirming critical scale thresholds for embodied reasoning. Compound Reasoning failures stem primarily from planning degradation (28.7\%), with models losing track of intermediate subgoals during execution.

Implicit Collaboration shows distinct timing failures (35.8\%)—models either initiate collaboration prematurely or miss coordination opportunities. This failure mode exhibits no scale correlation, indicating that collaboration timing demands reasoning mechanisms absent from current architectures. These failure patterns demonstrate that task categories stress fundamentally different cognitive capabilities, necessitating targeted architectural solutions beyond universal parameter scaling.

\subsection{Related Work}

\begin{table*}[htbp]
\footnotesize
\centering
\setlength{\tabcolsep}{2.5pt}
\begin{tabular*}{\textwidth}{@{\extracolsep{\fill}}lccccccc@{}}
\toprule
\textbf{Dataset} & \textbf{Scenes} & \textbf{Domain} & \textbf{Task Types} & \textbf{Actions} & \textbf{Action Space} & \textbf{Collab.} & \textbf{Auto Gen.} \\
\midrule
ALFRED & 120 & House & D & 7 & Static & --- & $\times$ \\
PARTNR & 60 & House & D & 11 & Static & Effic. & $\checkmark$ \\
BEHAVIOR-1K & 50 & Diverse & D,T & 6 & Static & --- & $\times$ \\
WAH & 7 & House & D & 10 & Static & Effic. & $\times$ \\
TDW-MAT & 6 & House & D,E & 7 & Static & Effic. & $\times$ \\
C-WAH & 6 & House & D,E & 7 & Static & Effic. & $\times$ \\
Overcooked & 5 & Kitchen & E,I,C & 6 & Static & Effic. & $\times$ \\
\midrule
\textbf{OmniEAR} & \textbf{1.5K} & \textbf{Diverse} & \textbf{D,A,T,R,E,I,C} & \textbf{218} & \textbf{Dynamic} & \textbf{Phys.} & \textbf{\checkmark} \\
\bottomrule
\end{tabular*}
\caption{Comparison of embodied AI datasets and benchmarks. Task types: D (Direct Command), A (Attribute Reasoning), T (Tool Use), R (Compound Reasoning), E (Explicit Collaboration), I (Implicit Collaboration), C (Compound Collaboration). Actions: number of available action types. Collab.: collaboration mechanism (Effic. = efficiency-based, Phys. = physical necessity-driven). Auto Gen.: automated task generation capability. Our framework uniquely combines comprehensive task coverage, dynamic action spaces, physical necessity-driven collaboration, and scalable automated generation.}
\label{tab:dataset_comparison}
\end{table*}

\label{sec:related_works}

\paragraph{Embodied Intelligence Benchmarks}
The embodied intelligence evaluation landscape has established diverse benchmark frameworks spanning navigation to complex manipulation tasks\citep{puig2023habitat,li2021igibson}.Table~\ref{tab:dataset_comparison} compares key characteristics across major embodied AI datasets.  ALFRED \citep{shridhar2020alfredbenchmarkinterpretinggrounded} provides foundational standards for instruction-following task evaluation, while BEHAVIOR-1K \citep{li2024behavior} extends coverage to 1,000 daily activity scenarios. These benchmarks effectively assess task execution capabilities, yet physical property modeling predominantly employs discrete state representations, such as binary door operations and object pickup/placement, with limited requirements for reasoning about continuous attributes including weight, hardness, and temperature. Our framework addresses this limitation by introducing continuous physical property reasoning tasks that require agents to compare object attributes and make decisions based on physical constraints.

\paragraph{Embodied Tool Use}
Tool usage evaluation in embodied AI exhibits stratified characteristics across different complexity levels. RoCo \citep{mandi2024roco} focuses on low-level manipulation skills such as grasping precision, while high-level benchmarks like PARTNR \citep{chang2024partnrbenchmarkplanningreasoning} adopt predefined tool configurations with agent action spaces fixed at task initialization. This design effectively simplifies evaluation complexity but presents limitations in assessing dynamic tool reasoning capabilities based on task requirements. Current approaches typically provide static tool sets, preventing evaluation of how agents should reason about capability gaps and tool acquisition needs. Our framework introduces dynamic tool acquisition mechanisms, requiring agents to autonomously infer tool requirements and expand their action spaces based on task demands, thereby supplementing existing evaluation dimensions.

\paragraph{Multi-Agent Collaboration}
Multi-agent embodied intelligence evaluation has emerged as a significant research direction, with related work achieving valuable progress in collaboration modeling\citep{sun2024aml,wang2024large}. PARTNR evaluates multi-agent planning capabilities through heterogeneous task design, TDW-MAT \citep{zhang2024buildingcooperativeembodiedagents} creates collaborative scenarios using load capacity constraints, and EmbodiedBench \citep{yang2025embodiedbench} focuses on task allocation and execution optimization. Existing approaches primarily model collaboration requirements through two pathways: explicit collaboration instructions that clearly specify inter-agent task division, and efficiency optimization that drives multi-agent participation to enhance task completion speed. However, real-world collaboration decisions often stem from physical constraints rather than external instructions or efficiency considerations. Our framework employs implicit collaboration design requiring agents to autonomously assess whether tasks exceed single-agent capability ranges based on physical constraints and determine collaboration strategies accordingly, transforming collaboration judgment from external instructions to constraint-driven internal reasoning processes.

\subsection{Hyperparameters}

\label{sec:hyperparameters}

\paragraph{Supervised Fine-Tuning.}
We performed full-parameter supervised fine-tuning on the \texttt{Qwen2.5-3B-Instruct} model to adapt it to our dataset. The training was conducted on 4x NVIDIA A100 GPUs. The effective batch size was 64, achieved through a per-device batch size of 1 and 16 gradient accumulation steps across 4 devices. Key hyperparameters for the SFT stage are summarized in Table~\ref{tab:sft_hyperparameters}.

\begin{table}[!h]
\centering
\begin{tabular}{@{}ll@{}}
\toprule
\textbf{Hyperparameter} & \textbf{Value} \\
\midrule
Base Model & \texttt{Qwen2.5-3B-Instruct} \\
Fine-tuning Method & Full-parameter \\
Effective Batch Size & 64 \\
Learning Rate & 1.0e-5 \\
LR Scheduler & Cosine Decay \\
Warmup Ratio & 0.1 \\
Training Epochs & 3 \\
Max Sequence Length & 15,360 \\
Precision & BF16 \\
\bottomrule
\end{tabular}
\caption{Hyperparameters for Supervised Fine-Tuning.}
\label{tab:sft_hyperparameters}
\end{table}

\paragraph{Model Inference.}
To ensure a fair and consistent comparison, all models were evaluated using the same set of inference parameters. We utilized the vLLM engine for efficient serving, with a tensor parallel size of 4. The decoding strategy was configured to balance response quality and exploration in complex reasoning tasks. The inference settings are detailed in Table~\ref{tab:inference_hyperparameters}.

\begin{table}[!h]
\centering
\begin{tabular}{@{}ll@{}}
\toprule
\textbf{Hyperparameter} & \textbf{Value} \\
\midrule
Inference Engine & vLLM \\
Tensor Parallel Size & 4 \\
Decoding Strategy & Nucleus Sampling \\
Temperature & 0.3 \\
Top-p & 1.0 (Default) \\
Max Generation Tokens & 4096 \\
Max Model Length & 15,360 \\
\bottomrule
\end{tabular}
\caption{Hyperparameters for Model Inference.}
\label{tab:inference_hyperparameters}
\end{table}


\subsection{Discussion}
\label{sec:discussion}

\paragraph{Embodied vs. Abstract Reasoning.}
Our results demonstrate that embodied reasoning requires distinct computational mechanisms from abstract reasoning in current language models. The persistent performance gaps across reasoning-specialized architectures indicate that chain-of-thought approaches cannot bridge the representational divide between symbolic manipulation and physical constraint processing. Current transformer architectures lack the specialized components necessary for grounding abstract representations in continuous physical properties.

\paragraph{Architectural Limitations.}
The constraint selection failures reveal that current attention mechanisms cannot dynamically filter task-relevant physical constraints from environmental noise. Unlike abstract reasoning tasks where all provided information typically bears relevance, embodied scenarios require selective attention over spatially and temporally distributed constraint sets. The discrete scaling transitions at 7B parameters indicate that embodied reasoning demands sufficient working memory capacity to simultaneously track environmental states, capability constraints, and coordination requirements—a computational bottleneck absent in pure language tasks.

\paragraph{Limitations and Future Work.}
Our text-based framework abstracts away continuous control, sensorimotor feedback, and real-time constraints present in physical embodied systems. While this abstraction enables systematic evaluation, it may not capture all aspects of embodied intelligence. The identified architectural requirements require validation in continuous control settings. Future work should investigate how these components integrate with sensorimotor processing and examine whether the observed computational bottlenecks persist in physically grounded systems. Additionally, exploring hybrid symbolic-neural architectures that can explicitly reason about physical laws while maintaining learned flexibility represents a promising direction\citep{rabinowitz2018machine}.
\subsection{Agent Prompt Configurations} \label{sec:appendix_prompts} 

This section details the system and user prompts used for different experimental configurations: single-agent and multi-agent scenarios.

\paragraph{Single-Agent Configuration.} 
This configuration tests individual agent reasoning capabilities through structured prompts.

\begin{tcolorbox}[
    colback=gray!10!white,
    colframe=black,
    title={System prompt for single-agent},
    fonttitle=\bfseries\small,
    breakable,
    enhanced,
    left=2mm,
    right=2mm,
    top=2mm,
    bottom=2mm,
    boxsep=1mm,
    arc=0pt,
    fontupper=\small\ttfamily,  % Changed from \tiny to \small
    before skip=6pt,
    after skip=6pt
]
\begin{flushleft}
\textbf{1. PRIMARY OBJECTIVE}\\
Your goal is to successfully complete the given task by systematically exploring the environment and interacting with objects. Success requires persistence, thorough exploration, and precise execution of interaction sequences.

\textbf{2. MANDATORY OUTPUT REQUIREMENTS}\\
You must follow these absolute rules in every single response:

\textbf{Strict Format Compliance}: Your entire output must be in the exact format `Thought: <reasoning>\textbackslash nAgent\_1\_Action: <command>`. Do not include any other text, explanations, or formatting.

\textbf{Command Validation}: The command you choose must be exactly as listed in the Available Actions provided in the user prompt. Do not invent or modify commands.

\textbf{Progress Verification}: After completing any part of the task, always re-read the task description in your next thought to verify if additional objectives remain incomplete.

\textbf{Completion Protocol}: Use the DONE action if and only if you have verified that all objectives in the task description have been successfully completed.

\textbf{3. OPERATIONAL FRAMEWORK}

\textbf{Exploration Strategy}: First use EXPLORE to thoroughly examine your current room. If the target isn't found, systematically GOTO and EXPLORE each unexplored room until completing the task.

\textbf{Interaction Sequence Protocol}: Always approach an object using GOTO before attempting any interaction with it. Always open containers using OPEN before taking items from or placing items into them. This sequence prevents interaction failures and ensures reliable task execution.

\textbf{4. CRITICAL FAILURE PATTERNS TO AVOID}

\textbf{Premature Task Abandonment}: Do not conclude failure without exploring every available room and container. Persistence is essential for task completion.

\textbf{Object Name Confusion}: Different names represent different objects. Verify exact matches between task requirements and available objects before taking action.

\textbf{Distance Interaction Violations}: Do not attempt to interact with objects that are not in immediate proximity. Always use GOTO to approach objects first.

\textbf{Container Access Oversight}: Do not forget to open containers before attempting to access their contents. This is a common cause of interaction failures.

\textbf{5. ERROR RECOVERY PROTOCOL}\\
If your chosen action results in an error, acknowledge the error in your next thought and immediately re-evaluate your strategy based on available information. Do not repeat failed actions unless the environmental situation has changed.

\textbf{6. REQUIRED OUTPUT FORMAT}\\
Your response must contain exactly two lines in this format:

Thought: [Your reasoning for taking this action]\\
Agent\_1\_Action: [Command from the available action list]

\textbf{Example Response}:\\
Thought: I am in the main work area and need to find the target objects. I have not explored the living room yet, so I should go there next.\\
Agent\_1\_Action: GOTO living\_room\_1
\end{flushleft}
\end{tcolorbox}

\begin{tcolorbox}[
    colback=gray!10!white,
    colframe=black,
    title={User prompt for single-agent},
    fonttitle=\bfseries\small,
    breakable,
    enhanced,
    left=2mm,
    right=2mm,
    top=2mm,
    bottom=2mm,
    boxsep=1mm,
    arc=0pt,
    fontupper=\small\ttfamily,  % Changed from \tiny to \small
    before skip=6pt,
    after skip=6pt
]
\begin{flushleft}
You are an intelligent agent tasked with completing the given objective by strictly following the operational framework established in your system instructions. Analyze the information provided below and determine the single best next action that will advance progress toward task completion.

\textbf{Current Environment}\\
\{environment\_description\}

\textbf{Task Objective}\\
\{task\_description\}

\textbf{Available Actions}\\
\{available\_actions\_list\}

\textbf{Recent Action History}\\
\{history\_summary\}

\textbf{Execution Guidelines}\\
Respond with exactly one thought and one action. Your thought should demonstrate systematic reasoning that considers the current situation, task requirements, and appropriate next steps. Your action must be selected from the available actions list and should represent the most logical progression toward completing the task objective.

Remember that systematic exploration, proper interaction sequences, and persistent problem-solving are essential for successful task completion. The available action descriptions will guide you on exactly how to execute each command effectively.
\end{flushleft}
\end{tcolorbox}

\paragraph{Multi-Agent Configuration.}
This configuration provides prompts for coordinated reasoning between two agents.

\begin{tcolorbox}[
    colback=gray!10!white,
    colframe=black,
    title={System prompt for multi-agent},
    fonttitle=\bfseries\small,
    breakable,
    enhanced,
    left=2mm,
    right=2mm,
    top=2mm,
    bottom=2mm,
    boxsep=1mm,
    arc=0pt,
    fontupper=\small\ttfamily,  % Changed from \tiny to \small
    before skip=6pt,
    after skip=6pt
]
\begin{flushleft}
You are a central coordination controller managing two intelligent agents working collaboratively to complete complex tasks. Your responsibility is to analyze the current situation, decompose objectives into executable subtasks, and assign optimal actions to both agents while ensuring efficient coordination and conflict avoidance.

\textbf{Core Coordination Principles}

\textbf{Strategic Assignment Protocol}: Assign actions based on each agent's current position, capabilities, and the optimal path toward task completion. Prioritize complementary actions that maximize overall efficiency.

\textbf{Conflict Prevention Framework}: Ensure that assigned actions do not create spatial conflicts, resource competition, or contradictory objectives between the two agents.

\textbf{Exploration Optimization}: When agents have completed their immediate objectives, prioritize exploration of unknown areas to gather additional environmental information and identify new opportunities for task advancement.

\textbf{Cooperation Command Protocol}

For collaborative tasks requiring joint action, implement the following cooperation strategy:

\textbf{Pre-Cooperation Positioning}: Before initiating any CORP\_ command sequence, ensure that both participating agents have successfully executed GOTO commands to reach the target object or designated cooperation zone.

\textbf{Cooperative Transport Sequence}: For tasks involving collaborative object movement, execute the following mandatory sequence without interruption:\\
1. CORP\_GRAB - Both agents grab/pick up the target object\\
2. CORP\_GOTO - Coordinated movement to the destination location\\
3. CORP\_PLACE - Synchronized placement of the object at the target location

\textbf{Critical CORP\_PLACE Requirement}: After executing CORP\_GOTO, you MUST execute CORP\_PLACE to actually place the object at the destination. The object is not considered "moved" until CORP\_PLACE is completed.

\textbf{Sequence Integrity Requirement}: The cooperative transport sequence must be executed continuously without interspersing other commands. Any interruption requires restarting the entire cooperation sequence. NEVER output DONE after CORP\_GOTO - always complete with CORP\_PLACE first.

\textbf{Cooperation Readiness Verification}: Verify that both agents are properly positioned and available for cooperation before initiating any CORP\_ command. This prevents coordination failures and ensures successful collaborative execution.

\textbf{Task Completion Management}

\textbf{Individual Agent Completion}: When an agent has no additional meaningful tasks to perform, assign the DONE command to that specific agent while continuing to provide actionable commands to the other agent.

\textbf{Final Task Termination}: The overall task concludes only when both agents simultaneously receive DONE commands, indicating that all objectives have been completed and no further actions are required.

\textbf{Continuation Protocol}: When one agent completes all its tasks, consistently assign DONE to that agent in all subsequent action assignments while continuing to provide meaningful actions to the remaining active agent until it also completes its objectives.

\textbf{Mandatory Output Format}

Your response must adhere to the following strict format without any additional content or explanations:

Thought: [Comprehensive analysis of current situation, task requirements, and strategic reasoning for action assignments]\\
Agent\_1\_Action: [Specific command for agent\_1 from available action set]\\
Agent\_2\_Action: [Specific command for agent\_2 from available action set]

Example:\\
Thought: Agent 1 is in the main work area and needs to explore, while agent 2 should go to the living room to find target items.\\
Agent\_1\_Action: EXPLORE\\
Agent\_2\_Action: GOTO living\_room\_1

\textbf{Strategic Planning Guidelines}

\textbf{Situational Assessment}: Evaluate each agent's current location, recent actions, and immediate objectives to determine the most effective next steps.

\textbf{Resource Allocation}: Consider the spatial distribution of tasks and assign agents to different areas when possible to maximize coverage and minimize redundancy.

\textbf{Progress Monitoring}: Track completion status of subtasks and adjust assignments based on evolving priorities and environmental discoveries.

\textbf{Efficiency Optimization}: Balance individual agent productivity with collaborative opportunities to achieve optimal overall task completion time.
\end{flushleft}
\end{tcolorbox}

\begin{tcolorbox}[
    colback=gray!10!white,
    colframe=black,
    title={User prompt for multi-agent},
    fonttitle=\bfseries\small,
    breakable,
    enhanced,
    left=2mm,
    right=2mm,
    top=2mm,
    bottom=2mm,
    boxsep=1mm,
    arc=0pt,
    fontupper=\small\ttfamily,  % Changed from \tiny to \small
    before skip=6pt,
    after skip=6pt
]
\begin{flushleft}
Analyze the provided information and generate coordinated action assignments for both agents:

\textbf{Current Environment State}\\
\{environment\_description\}

\textbf{Task Objectives}\\
\{task\_description\}

\textbf{Available Commands}\\
\{available\_actions\_list\}

\textbf{Agent Status and History}\\
\{history\_summary\}

\textbf{Coordination Requirements}\\
Generate action assignments that advance task completion while maintaining coordination efficiency. Ensure that cooperative tasks follow the established CORP\_ command protocols and that individual assignments complement overall strategic objectives.
\end{flushleft}
\end{tcolorbox} 